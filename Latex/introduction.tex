%%%%%%%%%%%%%%%%%%%%%%%%%%%%%%%%%%%%%%%%%
% Plik z wstępem do pracy
% Szablon pracy dyplomowej
% Wydział Informatyki 
% Zachodniopomorski Uniwersytet Technologiczny w Szczecinie
% autor Joanna Kołodziejczyk (jkolodziejczyk@zut.edu.pl)
% Bardzo wczesnym pierwowzorem szablonu był
% The Legrand Orange Book
% Version 5.0 (29/05/2025)
%
% Modifications to LOB assigned by %JK
%%%%%%%%%%%%%%%%%%%%%%%%%%%%%%%%%%%%%%%%%


\chapter*{Wstęp}
\index{Wstęp}

\section*{Wprowadzenie}
\index{Wprowadzenie}

We współczesnym świecie, który staje się coraz bardziej cyfrowy, powstaje coraz większe zapotrzebowanie na nowoczesne systemy informatyczne.
Rozwój technologii informacyjnych i komunikacyjnych (ICT)\footnote{ICT - Information and Communications Technology} wspiera dynamiczny rozwój gospodarki oraz umożliwia innowacje w rożnych dziedzinach oraz codzienny życiu.
Wraz z dynamicznym rozwojem technologii webowych pojawiają się nowe możliwości tworzenia nowoczesnych, responsywnych oraz wydajnych aplikacji wspierających procesy biznesowe.

W niniejszej pracy dyplomowej koncentruję się na projektowaniu, doborze technologii z coraz to większego wachlarza dostępnych frameworków \footnote{Framework albo platforma programistyczna – szkielet do budowy aplikacji. Definiuje on strukturę aplikacji oraz ogólny mechanizm jej działania, a także dostarcza zestaw komponentów i bibliotek ogólnego przeznaczenia do wykonywania określonych zadań. Programista tworzy aplikację, rozbudowując i dostosowując poszczególne komponenty do wymagań realizowanego projektu, tworząc w ten sposób gotową aplikację. } oraz implementacji aplikacji wykonanej w technologii web.

Projekt został zrealizowany w oparciu o framework Angular w części frontendowej, ASP.NET Core dla backendu oraz bazę danych SQLite.

\section*{Cel pracy}
\index{Cel pracy}

Głównym celem pracy jest zaprojektowanie i wdrożenie kompleksowej aplikacji webowej do tworzenia interaktywnego atlasu ptaków, które spełnia wymagania stawiane nowoczesnym systemom oraz aktualne standardy branży, uwzględniając aspekty takie jak:
\begin{itemize}
    \item responsywność i dostępność na różnych urządzeniach,
    \item wydajność i optymalizacja działania,
    \item bezpieczeństwo danych użytkowników,
    \item łatwość w utrzymaniu i rozwijaniu aplikacji.
\end{itemize}

\section*{Zakres pracy}
\index{Zakres pracy}

Praca obejmuje następujące etapy: z części teoretycznej prezentującej zagadnienie aplikacji webowej oraz dobór odpowiednich technologii i narzędzi oraz projekt systemu,
z części praktycznej, która obejmuje implementacje poszczególnych komponentów aplikacji oraz wdrożenie systemu na serwerze produkcyjnym.

Warte podkreślenia jest, że realizacja projektu stanowi praktyczne zastosowanie teorii z zakresu inżynierii oprogramowania oraz zdobytej wiedzy i doświadczenia na studiach
oraz w trakcie praktyk zawodowych i pracy w branży IT. Ponadto, projekt ten stanowi przykład zastosowania nowoczesnego stacku technologicznego, którego znajomość przez specjalistów IT jest coraz bardziej ceniona i pożądana w branży.

\section*{Struktura pracy}
\index{Struktura pracy}

Praca jest podzielona na następujące rozdziały, które skupiają się na poszczególnych etapach tworzenia systemu:

\begin{enumerate}
	\item Analiza wymagań i założeń projektowych.
	\item Wybór i uzasadnienie technologii
	\item Projekt
	\item Implementacja
	\item Wdrożenie
\end{enumerate}

% Wstęp powinien być nie dłuższy niż 2 strony. Najlepiej napisać go dopiero, gdy praca jest już skończona i wszystkie jej części spisane.

% Wstęp powinien zawierać:

% \begin{enumerate}
% \item Opis dziedziny jakiej dotyczy praca, ze wskazaniem, że temat pracy jest ważny, bieżący, itp.
% \item Jaki problem z dziedziny się rozwiązuje.
% \item Cel i teza pracy
% \item W jaki sposób cel zostanie osiągnięty a tez potwierdzona.
% \item Struktura pracy.
% \end{enumerate} 