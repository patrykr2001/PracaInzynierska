%%%%%%%%%%%%%%%%%%%%%%%%%%%%%%%%%%%%%%%%%
% Szablon pracy dyplomowej
% Wydział Informatyki 
% Zachodniopomorski Uniwersytet Technologiczny w Szczecinie
% autor Joanna Kołodziejczyk (jkolodziejczyk@zut.edu.pl)
% Bardzo wczesnym pierwowzorem szablonu był
% The Legrand Orange Book
% Version 5.0 (29/05/2025)
%
% Modifications to LOB assigned by %JK
%%%%%%%%%%%%%%%%%%%%%%%%%%%%%%%%%%%%%%%%%

%----------------------------------------------------------------------------------------
%	CHAPTER 1
% 	author: Joanna Kolodziejczyk (jkolodziejczyk@zut.edu.pl)
%----------------------------------------------------------------------------------------

\chapter{Analiza wymagań i założeń projektowych}
\label{rozdzial1}

\section{Analiza potrzeb użytkowników}
\index{Analiza potrzeb użytkowników}

\subsection{Identyfikacja grupy docelowej}
\index{Identyfikacja grupy docelowej}

Jednym z najważniejszych etapów w projektowaniu systemu informatycznego jest identyfikacja grupy docelowej użytkowników.
Po analizie rynku i dostępnych rozwiązań można wyróżnić kilka kluczowych grup użytkowników:
\begin{itemize}
	\item \textbf{Profesjonalni Ornitolodzy} - osoby zajmujące się nauką i badaniem ptaków zawodowo, główna grupa użytkowników wykorzystujących systemy informatyczne do gromadzenia i analizy danych.
	\item \textbf{Amatorzy i hobbyści} - grupa która interesuje się ptakami w wolnym czasie, lub osoby które chcące rozwijać swoją wiedzę. Często korzystają z aplikacji do identyfikacji lub dokumentacji napotkanych ptaków.,
	\item \textbf{Studenci kierunków przyrodniczych} - studenci biologii, ekologii lub innych kierunków przyrodniczych, którzy mogą korzystać z systemów do nauki i badań terenowych.
	\item \textbf{Badacze i naukowcy} - grupa zbliżona zawodowo do profesjonalnych ornitologów, jednak często badająca inne aspekty związane z ptakami i środowiskiem życia.
\end{itemize}

\subsection{Role i uprawnienia użytkowników}
\index{Role i uprawnienia użytkowników}

W systemach informatycznych istotnym aspektem jest zarządzanie rolami i uprawnieniami użytkowników. W przypadku omawianego systemu można wyróżnić następujące role:
\begin{itemize}
	\item \textbf{Administrator} - osoba odpowiedzialna za zarządzanie systemem, użytkownikami, konfiguracje, utrzymanie.
	\item \textbf{Użytkownik} - osoba korzystająca z systemu do identyfikacji ptaków, dokumentacji obserwacji i przeglądania dostępnych zasobów.
	\item \textbf{Moderator} - osoba odpowiedzialna za sprawdzanie i moderowanie treści wprowadzanych przez użytkowników.
\end{itemize}

\subsection{Profil technologiczny użytkowników}
\index{Profil technologiczny użytkowników}

Użytkownicy charakteryzują się zróżnicowanym poziomem zaawansowania technologicznego, co powinno zostać uwzględnione w procesie projektowym. Interfejsu powinny być proste w obsłudze oraz czytelne.

Spora część użytkowników wciąż głównie korzysta z komputerów osobistych do obsługi systemów biznesowych - czy to w domu czy w miejscu pracy. Jednak nie można zapomnieć o urządzeniach mobilnych, które coraz bardziej zastępują nam komputery w codziennym życiu. Interfejs systemu powinien być dostosowany zarówno dla dużych ekranów komputerowych jak i do małych, wąskich i podłużnych wyświetlaczy telefonów.

\section{Specyfikacja funkcjonalna}
\index{Specyfikacja funkcjonalna}

\section{Wymagania niefunkcjonalne}
\index{Wymagania niefunkcjonalne}

\section{Analiza podobnych rozwiązań na rynku}
\index{Analiza podobnych rozwiązań na rynku}