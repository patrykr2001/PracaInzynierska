%%%%%%%%%%%%%%%%%%%%%%%%%%%%%%%%%%%%%%%%%
% Szablon pracy dyplomowej
% Wydział Informatyki 
% Zachodniopomorski Uniwersytet Technologiczny w Szczecinie
% autor Joanna Kołodziejczyk (jkolodziejczyk@zut.edu.pl)
% Bardzo wczesnym pierwowzorem szablonu był
% The Legrand Orange Book
% Version 5.0 (29/05/2025)
%
% Modifications to LOB assigned by %JK
%%%%%%%%%%%%%%%%%%%%%%%%%%%%%%%%%%%%%%%%%

%----------------------------------------------------------------------------------------
%	CHAPTER 1
% 	author: Patryk Rakowski (rp51626@zut.edu.pl)
%----------------------------------------------------------------------------------------

\chapter{Analiza wymagań i założeń projektowych}
\label{rozdzial1}

\section{Analiza potrzeb użytkowników}
\index{Analiza potrzeb użytkowników}

\subsection{Identyfikacja grupy docelowej}
\index{Identyfikacja grupy docelowej}

Jednym z najważniejszych etapów w projektowaniu systemu informatycznego jest identyfikacja grupy docelowej użytkowników.
Po analizie rynku i dostępnych rozwiązań można wyróżnić kilka kluczowych grup użytkowników:
\begin{itemize}
	\item \textbf{Profesjonalni Ornitolodzy} - osoby zajmujące się nauką i badaniem ptaków zawodowo, główna grupa użytkowników wykorzystujących systemy informatyczne do gromadzenia i analizy danych.
	\item \textbf{Amatorzy i hobbyści} - grupa która interesuje się ptakami w wolnym czasie, lub osoby które chcące rozwijać swoją wiedzę. Często korzystają z aplikacji do identyfikacji lub dokumentacji napotkanych ptaków.,
	\item \textbf{Studenci kierunków przyrodniczych} - studenci biologii, ekologii lub innych kierunków przyrodniczych, którzy mogą korzystać z systemów do nauki i badań terenowych.
	\item \textbf{Badacze i naukowcy} - grupa zbliżona zawodowo do profesjonalnych ornitologów, jednak często badająca inne aspekty związane z ptakami i środowiskiem życia.
\end{itemize}

\subsection{Role i uprawnienia użytkowników}
\index{Role i uprawnienia użytkowników}

W systemach informatycznych istotnym aspektem jest zarządzanie rolami i uprawnieniami użytkowników. W przypadku omawianego systemu można wyróżnić następujące role:
\begin{itemize}
	\item \textbf{Administrator} - osoba odpowiedzialna za zarządzanie systemem, użytkownikami, konfiguracje, utrzymanie.
	\item \textbf{Użytkownik} - osoba korzystająca z systemu do identyfikacji ptaków, dokumentacji obserwacji i przeglądania dostępnych zasobów.
	\item \textbf{Moderator} - osoba odpowiedzialna za sprawdzanie i moderowanie treści wprowadzanych przez użytkowników.
\end{itemize}

\subsection{Profil technologiczny użytkowników}
\index{Profil technologiczny użytkowników}

Użytkownicy charakteryzują się zróżnicowanym poziomem zaawansowania technologicznego, co powinno zostać uwzględnione w procesie projektowym. Interfejsu powinny być proste w obsłudze oraz czytelne.

Spora część użytkowników wciąż głównie korzysta z komputerów osobistych do obsługi systemów biznesowych - czy to w domu czy w miejscu pracy. Jednak nie można zapomnieć o urządzeniach mobilnych, które coraz bardziej zastępują nam komputery w codziennym życiu. Interfejs systemu powinien być dostosowany zarówno dla dużych ekranów komputerowych jak i do małych, wąskich i podłużnych wyświetlaczy telefonów.

\section{Specyfikacja funkcjonalna}
\index{Specyfikacja funkcjonalna}

Celem projektu jest stworzenie aplikacji internetowej umożliwiającej użytkownikom dodawanie ptaków, obserwacji i przeglądanie dostępnych treści. System ma za zadanie wspierać amatorów jak i zaawansowanych ornitologów i badaczy w dokumentowaniu i analizie występowania różnych gatunków ptaków w Polsce.
Poniżej przedstawiono wymagania funkcjonalne systemu:

\subsection{Użytkownicy}
\begin{itemize}
	\item \textbf{Rejestracja i logowanie użytkowników} - system umożliwia rejestracje nowych użytkowników oraz logowanie do aplikacji.
	\item \textbf{Zarządzenie użytkownikami} - system umożliwia administratorom zarządzanie kontami użytkowników.
	\item \textbf{Zarządzanie swoim kontem} - system umożliwia użytkownikom zarządzanie swoim kontem: zmianę adresu e-mail, hasła oraz dezaktywacje konta.
\end{itemize}

\subsection{Strona główna i mapa}
\begin{itemize}
	\item \textbf{Mapa obserwacji} - system udostępnia mapę obserwacji ptaków z zaznaczonymi poszczególnymi obserwacjami.
	\item \textbf{Filtrowanie danych} - mapa umożliwia filtrowanie obserwacji po roku, tygodniu oraz gatunkach ptaków.
	\item \textbf{Szczegóły obserwacji} - znaczniki na mapie przenoszą użytkowników do strony obserwacji której dotyczą.
\end{itemize}

\subsection{Obserwacje}
\begin{itemize}
	\item \textbf{Strona obserwacji} - system udostępnia użytkownikom stronę z zweryfikowanymi obserwacjami ptaków.
	\item \textbf{Filtrowanie danych} - system umożliwia filtrowanie danych na stronie obserwacji po zakresie dat oraz gatunkach ptaków.
	\item \textbf{Dodawanie nowych obserwacji} - system umożliwia dodawanie nowych obserwacji niezweryfikowanych zarejestrowanym użytkownikom. Obserwacja składa się z następujących danych:
	\begin{itemize}
		\item wybór ptaka,
		\item data obserwacji,
		\item liczba osobników,
		\item warunki pogodowe,
		\item siedlisko,
		\item opis,
		\item lokalizacja geograficzna,
		\item zdjęcia.
	\end{itemize}
	\item \textbf{Strona szczegółów obserwacji} - system udostępnia dla każdej obserwacji stronę ze szczegółami danej obserwacji.
		Moderator lub użytkownik który dodał daną obserwację może edytować jej dane lub ją usunąć.
	\item \textbf{Moderacja obserwacji} - strona obserwacji udostępnia moderatorom zakładkę z obserwacjami niezweryfikowanymi oraz możliwość weryfikacji danej obserwacji poprzez podstronę szczegółów obserwacji.
\end{itemize}

\subsection{Ptaki}
\begin{itemize}
	\item \textbf{Strona ptaków} - system udostępnia użytkownikom stronę z zweryfikowanymi ptaków.
	\item \textbf{Filtrowanie danych} - system umożliwia wyszukiwanie ptaków po:
		\begin{itemize}
			\item nazwie,
			\item nazwie naukowej,
			\item rodzinie,
			\item rzędzie,
			\item rodzajowi,
			\item gatunkowi.
		\end{itemize}
	\item \textbf{Dodawanie nowych ptaków} - system umożliwia dodawanie nowych ptaków niezweryfikowanych zarejestrowanym użytkownikom. Ptak składa się z następujących danych:
	\begin{itemize}
		\item nazwa,
		\item nazwa naukowa,
		\item zdjęcie,
		\item rodzina,
		\item rząd,
		\item rodzaj,
		\item gatunek,
		\item status ochrony,
		\item siedlisko,
		\item dieta,
		\item rozmiar,
		\item waga,
		\item rozpiętość skrzydeł,
		\item długość życia,
		\item sezon lęgowy,
		\item opis.
	\end{itemize}
	\item \textbf{Strona szczegółów ptaka} - system udostępnia dla każdego ptaka stronę ze szczegółami danego ptaka.
		Moderator lub użytkownik który dodał danego ptaka może edytować jego dane lub go usunąć.
	\item \textbf{Moderacja ptaków} - strona ptaków udostępnia moderatorom zakładkę z ptakami niezweryfikowanymi oraz możliwość weryfikacji danego ptaka poprzez podstronę szczegółów obserwacji.
\end{itemize}

\subsection{Statystyki}
System udostępnia stronę statystyk obserwacji ptaków. Strona zawiera informację takie jak:
\begin{itemize}
	\item \textbf{Panel główny} z kluczowymi wskaźnikami:
		\begin{itemize}
			\item całkowita liczba zaobserwowanych ptaków,
			\item liczba różnych gatunków ptaków,
			\item najczęściej spotykane gatunki,
			\item najrzadziej spotykane gatunki.
		\end{itemize}
	\item \textbf{Wykresy i wizualizacje}
		\begin{itemize}
			\item mapa cieplna pokazująca najczęstsze lokalizacje obserwacji,
			\item wykres liniowy pokazujący trendy obserwacji w czasie.
		\end{itemize}
\end{itemize}

\section{Wymagania niefunkcjonalne}
\index{Wymagania niefunkcjonalne}

\section{Analiza podobnych rozwiązań na rynku}
\index{Analiza podobnych rozwiązań na rynku}