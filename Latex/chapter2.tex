%%%%%%%%%%%%%%%%%%%%%%%%%%%%%%%%%%%%%%%%%
% Szablon pracy dyplomowej
% Wydział Informatyki 
% Zachodniopomorski Uniwersytet Technologiczny w Szczecinie
% autor Joanna Kołodziejczyk (jkolodziejczyk@zut.edu.pl)
% Bardzo wczesnym pierwowzorem szablonu był
% The Legrand Orange Book
% Version 5.0 (29/05/2025)
%
% Modifications to LOB assigned by %JK
%%%%%%%%%%%%%%%%%%%%%%%%%%%%%%%%%%%%%%%%%


%----------------------------------------------------------------------------------------
%	CHAPTER 2
%----------------------------------------------------------------------------------------
\chapter{Wybór i uzasadnienie technologii}
\label{rozdzial2}

\section{Analiza dostępnych rozwiązań technologicznych}
\index{Analiza dostępnych rozwiązań technologicznych}

\section{Uzasadnienie wyboru Angular jako frameworka frontendowego}
\index{Uzasadnienie wyboru Angular jako frameworka frontendowego}

\section{Uzasadnienie wyboru ASP.NET Core jako backendu}
\index{Uzasadnienie wyboru ASP.NET Core jako backendu}

\section{Uzasadnienie wyboru SQLite jako bazy danych}
\index{Uzasadnienie wyboru SQLite jako bazy danych}

\section{Omówienie innych wykorzystanych technologii}
\index{Omówienie innych wykorzystanych technologii}
%  (Leaflet, Material Design)


% \chaptermark{Elementy w pracy} % Tekst, który wyświetli się w nagłówku strony,  jeżeli jest za długi tytuł rozdziału

% Szablon definiuje różne otoczenia, z których może korzystać autor pracy dyplomowej. Takowe zostały opisane w niniejszym rozdziale. Jeżeli otoczenie jest zawarte w szablonie, to właśnie takie należy stosować. Jeżeli w ramach pracy zachodzi potrzeba korzystania z elementów nie ujętych w szablonie można je stosować dowolnie.

% \section{Listy}

% Strukturę list wyliczeniowych poprawnie wykorzystywanych w języku polskim opisują słowniki. Oto fragment z poradnika \url{sjp.pwn.pl}. Podpunkty zdaniowe powinny się zaczynać wielką literą, a niezdaniowe małą. Jeśli podpunkt jest zdaniem, to zamykamy go kropką lub znakiem równoważnym, jeśli nie – przecinkiem lub średnikiem.  Nagłówek listy powinien się kończyć dwukropkiem nawet wtedy, gdy następują po nim podpunkty zdaniowe, pisane wielką literą. 

% \noindent Listy numerowane wykorzystuje się, gdy kolejność elementów jest istotna:
% \begin{enumerate}
% \item element 1,
% \item element 2,
% \item element 34234.
% \end{enumerate}

% \noindent Listy wypunktowane wykorzystuje się, gdy kolejność elementów jest istotna:
% \begin{itemize}
% \item osoba Alicja A, 
% \item zwierzak kot Alicji,
% \item dom Alicji.
% \end{itemize}

% Jeżeli zamiast punktora lub wyliczenia istnieje potrzeba wykorzystania innego słowa jako wyliczenia,to możliwe jest stosowanie listy z nagłówkami (descriptions-definitions),
% \begin{description}
% \item[Nazwa] Opis;
% \item[Słowo] Definicja;
% \item[Komentarz] Wywód.
% \end{description}

% \section{Tabele i rysunki w pracy}

% Tabele podpisuje się z góry. Ważne, że każdy element osadzony w treści pracy musi zostać przywołany w tekście np. w Tabeli: \ref{tab:eksperyment1} zaprezentowano wyniki eksperymentu

% \begin{table}[ht]
% \centering
% \caption{Tabelka z wynikami eksperymentu 1} 
% \begin{tabular}{l l l}
% \toprule
% \textbf{Lek} & \textbf{Odpowiedź 1} & \textbf{Odpowiedź 2}\\
% \midrule
% Lek 1 & 0.0003262 & 0.562 \\
% Lek 2 & 0.0015681 & 0.910 \\
% Lek 3 & 0.0009271 & 0.296 \\
% \bottomrule
% \end{tabular}
% \label{tab:eksperyment1} % Etykiety dla każdej tabeli musi być unikalna
% \end{table}

% Rysunki podpisywane są z dołu. Należy tak jak w tabeli w treści pracy odnieść się do etykiety rysunku, np. Rysunek \ref{fig:rys1} przedstawia

% \begin{figure}[!ht]
% 		\centering
% 		%\vspace{2mm}
% 		\includegraphics[scale=0.6]{Pictures/placeholder.jpg}
% 		\caption{To jest przykład osadzania rysunków (źródło:), który może mieć długi podpis nawet na 10000000 00000 000000 00000 linijek.. 10000000 00000 000000 00000 linijek. 10000000 00000 000000 00000 linijek.}
% 		\label{fig:rys1}
% \end{figure}



% \section{Twierdzenia i definicje}

% W pracach o rozbudowanej części teoretycznej może istnieć potrzeba prezentowania twierdzeń i definicji. Poniżej przykłady stosowania odpowiednich otoczeń. 

% A oto twierdzenie składające się z kilku równań.

% \begin{theorem}[Tytuł twierdzenia]
% In $E=\mathbb{R}^n$ all norms are equivalent. It has the properties:
% \begin{align}
% & \big| ||\mathbf{x}|| - ||\mathbf{y}|| \big|\leq || \mathbf{x}- \mathbf{y}||\\
% &  ||\sum_{i=1}^n\mathbf{x}_i||\leq \sum_{i=1}^n||\mathbf{x}_i||\quad\text{where $n$ is a finite integer}
% \end{align}
% \end{theorem}

% Jest to twierdzenie składające się z tylko jednej linii.

% \begin{theorem}
% Zbiór $\mathcal{D}(G)$ ma gęstość $L^2(G)$, $|\cdot|_0$. 
% \end{theorem}


% To jest przykład definicji. Definicja może być matematyczna lub może definiować koncepcję. Jeżeli w pracy wprowadza się nowe pojęcia najlepiej ująć je w formie definicji, co będzie spójnie akcentować taki fakt.

% \begin{definition}[Tytuł definicji]
% Given a vector space $E$, a norm on $E$ is an application, denoted $||\cdot||$, $E$ in $\mathbb{R}^+=[0,+\infty[$ such that:
% \begin{align}
% & ||\mathbf{x}||=0\ \Rightarrow\ \mathbf{x}=\mathbf{0}\\
% & ||\lambda \mathbf{x}||=|\lambda|\cdot ||\mathbf{x}||\\
% & ||\mathbf{x}+\mathbf{y}||\leq ||\mathbf{x}||+||\mathbf{y}||
% \end{align}
% \end{definition}


% %------------------------------------------------

% \section{Przykłady}

% W wielu pracach do przedstawionych idei prezentuje się przykłady. Na tą okoliczność przygotowano otoczenie {\it example}, które pomoże utrzymać spójny wygląd wszystkich przykładów. 

% Przykład może dotyczyć obliczeń lub prezentacji działania np. metody, algorytmu kodu, itp, itd.

% \begin{example}

% Let $G=\{x\in\mathbb{R}^2:|x|<3\}$ and denoted by: $x^0=(1,1)$; consider the function:
% \begin{equation}
% f(x)=\left\{\begin{aligned} & \mathrm{e}^{|x|} & & \text{si $|x-x^0|\leq 1/2$}\\
% & 0 & & \text{si $|x-x^0|> 1/2$}\end{aligned}\right.
% \end{equation}
% The function $f$ has bounded support, we can take $A=\{x\in\mathbb{R}^2:|x-x^0|\leq 1/2+\epsilon\}$ for all $\epsilon\in\intoo{0}{5/2-\sqrt{2}}$.
% \end{example}


% \begin{example}[Przykład krzyżowania]
% \begin{align*}
% &\text{Pokolenie rodziców:} & & & &\text{Pokolenie potomków:} \\
% &x_1 = (\textbf{00110} | \textbf{011}) & &\xrightarrow{\text{krzyżowanie}} & &(\textbf{00110}|1 0 1)\\
% &x_2 = (0 1 1 0 1 | 1 0 1) & & & &(0 1 1 0 1|\textbf{011})\\
% \end{align*}
% \end{example}


% %------------------------------------------------

% \section{Kod źródłowy}

% Jeżeli w pracy prezentowany jest kod źródłowy, to należy skorzystać z predefiniowanego i skonfigurowanego (w structure.tex) otoczenia lstlisting. Jeżeli autor chciałby umieścić w pracy pseudokod, to może posłużyć się wybranym przez siebie otoczeniem.

% Zasada odwołania w tekście do algorytmu jest taka sama jak do tabeli i rysunku np. algorytm \ref{alg:kod1} został opracowany na potrzeby...

% \begin{lstlisting}[language=Python, caption=Fragment algorytmu xxx, label=alg:kod1]
% import numpy as np
% # change this value for a different result
% celsius = 37.5

% # calculate fahrenheit
% fahrenheit = (celsius * 1.8) + 32
% print('%0.1f degree Celsius is equal to %0.1f degree Fahrenheit' %(celsius,fahrenheit))
 
% def incmatrix(genl1,genl2):
%     m = len(genl1)
%     n = len(genl2)
%     M = None #to become the incidence matrix
%     VT = np.zeros((n*m,1), int)  #zmienna ąźćł
 
%     #compute the bitwise xor matrix
%     M1 = bitxormatrix(genl1)
%     M2 = np.triu(bitxormatrix(genl2),1) 
 
%     for i in range(m-1):
%         for j in range(i+1, m):
%             [r,c] = np.where(M2 == M1[i,j])
%             for k in range(len(r)):
%                 VT[(i)*n + r[k]] = 1;
%                 VT[(i)*n + c[k]] = 1;
%                 VT[(j)*n + r[k]] = 1;
%                 VT[(j)*n + c[k]] = 1;
 
%                 if M is None:
%                     M = np.copy(VT)
%                 else:
%                     M = np.concatenate((M, VT), 1)
 
%     return M
% \end{lstlisting}
