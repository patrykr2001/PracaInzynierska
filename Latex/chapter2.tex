%%%%%%%%%%%%%%%%%%%%%%%%%%%%%%%%%%%%%%%%%
% Szablon pracy dyplomowej
% Wydział Informatyki 
% Zachodniopomorski Uniwersytet Technologiczny w Szczecinie
% autor Joanna Kołodziejczyk (jkolodziejczyk@zut.edu.pl)
% Bardzo wczesnym pierwowzorem szablonu był
% The Legrand Orange Book
% Version 5.0 (29/05/2025)
%
% Modifications to LOB assigned by %JK
%%%%%%%%%%%%%%%%%%%%%%%%%%%%%%%%%%%%%%%%%


%----------------------------------------------------------------------------------------
%	CHAPTER 2
%----------------------------------------------------------------------------------------
\chapter{Wybór i uzasadnienie technologii}
\label{rozdzial2}

W tej sekcji zostało przedstawione przegląd i porównanie dostępnych technologii webowych oraz uzasadniony ich wybór w implementacji systemu.

\section{Analiza dostępnych rozwiązań technologicznych}
\index{Analiza dostępnych rozwiązań technologicznych}

\subsection{Frameworki frontendowe}
Poniżej zostało przedstawione porównanie głównych frameworków frontendowych.\cite{reactvsvuevsangularvsnextjs}

\subsubsection{Angular}
Kompletny framework utrzymywany przez Google. Angular wymusza strukture z dużym naciskiem na TypeScript, RxJS i skalowalną architekturę.
\begin{itemize}
	\item \textbf{Język} - TypeScript,
	\item \textbf{Kluczowe cechy} - wstrzykiwanie zależności\footnote{Wstrzykiwanie zależności – wzorzec projektowy i wzorzec architektury oprogramowania polegający na usuwaniu bezpośrednich zależności pomiędzy komponentami na rzecz architektury typu plug-in. Polega na przekazywaniu gotowych, utworzonych instancji obiektów udostępniających swoje metody i właściwości obiektom, które z nich korzystają.}, RxJS\footnote{RxJS to biblioteka do programowania reaktywnego przy użyciu Observables, aby ułatwić tworzenie kodu asynchronicznego lub opartego na wywołaniach zwrotnych. Z ang. \cite{rxjs.dev}}, routing, testowanie, CLI\footnote{CLI to skrót od „Command Line Interface” (interfejs wiersza poleceń). Jest to sposób interakcji użytkownika z komputerem poprzez wprowadzanie poleceń tekstowych w terminalu lub wierszu poleceń, zamiast korzystania z interfejsu graficznego (GUI).},
	\item \textbf{Przypadki użycia} - Aplikacje dla przedsiębiorstw, narzędzia wewnętrzne, sektory regulowane.
\end{itemize}

Zalety:
\begin{itemize}
	\item Kompleksowe rozwiązanie,
	\item Wbudowany system zarządzania stanem (RxJS),
	\item Sile typowanie dzięki wykorzystaniu TypeScript,
	\item Oficjalna biblioteka komponentów (Material),
	\item Dependency injection wbudowany w framework,
	\item Oficjalne wsparcie od Google,
	\item Długoterminowe wsparcie (LTS),
	\item Bogata dokumentacja i duża społeczność.
\end{itemize}

Wady:
\begin{itemize}
	\item Wysoki próg wejścia,
	\item Większy rozmiar aplikacji w porównaniu do innych frameworków,
	\item Mniej elastyczny niz React,
	\item Wymaga znajomości wielu konceptów inżynierii oprogramowania,
	\item Mniejsza ilość dostępnych komponentów niż w React.
\end{itemize}

\subsubsection{React}
Biblioteka UI opracowana przez firmę Meta. React skupia się wyłącznie na warstwie widoku, dając elastyczność w wyborze własnych narzędzi do routingu, zarządzania stanem i budowania.

\begin{itemize}
	\item \textbf{Język} - JavaScript lub TypeScript (z JSX\footnote{JSX (JavaScript XML) – format zapisu kodu HTML oraz XML wewnątrz języka JavaScript. Pierwotnie zaproponowany przez Facebooka i użyty w bibliotece React.js. Korzystają z niego także inne biblioteki.}),
	\item \textbf{Kluczowe cechy} - hooki, wirtualny DOM\footnote{Obiektowy model dokumentu – sposób reprezentacji złożonych dokumentów XML i HTML w postaci modelu obiektowego. Model ten jest niezależny od platformy i języka programowania.}, architektura oparta na komponentach,
	\item \textbf{Przypadki użycia} - wszelkiego rodzaju interfejsy użytkownika, od prostych komponentów po aplikacje wieloplatformowe.
\end{itemize}

Zalety:
\begin{itemize}
	\item Elastyczność i swoboda w wyborze narzędzi,
	\item Duża ilość dostępnych komponentów i bibliotek,
	\item Mniejszy próg wejścia i szybsza nauka,
	\item Virtual DOM dla lepszej wydajności,
	\item Popularny z dużą społecznością,
	\item Wsparcie od Meta (Facebook).
\end{itemize}

Wady:
\begin{itemize}
	\item Wymaga dodatkowych bibliotek do pełnej funkcjonalności,
	\item Brak wbudowanego systemu zarządzania stanem,
	\item Konieczność samodzielnego wyboru narzędzi.
\end{itemize}

\subsubsection{Vue}
Progresywny framework opracowany przez Evana You. Vue kładzie nacisk na prostotę i wydajność, w wbudowaną mocą reaktywności i kompozycji.

\begin{itemize}
	\item \textbf{Język} - JavaScript/TypeScript z opartymi na HTML\footnote{HTML – język znaczników stosowany do tworzenia dokumentów hipertekstowych.} szablonami,
	\item \textbf{Kluczowe cechy} - API kompozycji, SFC\footnote{Vue Single-File Components(SFC) - Komponenty jednoplikowe Vue}, silnik reaktywności,
	\item \textbf{Przypadki użycia} - pulpity nawigacyjne, panele administracyjne, CMS-y\footnote{Content Management System (CMS) - System zarządzania treścią – oprogramowanie pozwalające na łatwe utworzenie i prowadzenie serwisu WWW, a także jego późniejszą aktualizację i rozbudowę, również przez redakcyjny personel nietechniczny.} i SPA\footnote{Single Page Application (SPA) - jednostronicowa aplikacja internetowa. Rodzaj aplikacji webowej, która działa w przeglądarce użytkownika i nie wymaga przeładowywania strony przy przechodzeniu między sekcjami}.
\end{itemize}

Zalety:
\begin{itemize}
	\item Niski próg wejścia,
	\item Elastyczność implementacji,
	\item Bardzo dobra dokumentacja,
	\item Mniejszy rozmiar aplikacji,
	\item Prosty system komponentów.
\end{itemize}

Wady:
\begin{itemize}
	\item Mniejsza społeczność niż w przypadku innych frameworków,
	\item Mniej dostępnych komponentów,
	\item Mniej popularny w profesjonalnych zastosowaniach,
	\item Brak oficjalnego wsparcia dużej firmy.
\end{itemize}

\subsubsection{Next.js}
Oparty na React framework full-stack zbudowany przez firmę Vercel. Next.js przekształca Reacta w platformę do budowania aplikacji SSR\footnote{Server Side Rendering (SSR) - renderowanie po stronie serwera, to technika generowania strony internetowej na serwerze jeszcze przed wysłaniem jej do klienta. Jest to alternatywa dla Client Side Rendering (CSR)}, SSG\footnote{SSG (Static Site Generation) - technika tworzenia stron internetowych, gdzie całość zawartości strony generowana jest w momencie jej budowy, a następnie serwowana jako statyczne pliki HTML.} oraz ESR\footnote{Edge Side Rendering (ESR) - technika renderowania stron internetowych, która polega na wstępnym renderowaniu stron internetowych na obrzeżach sieci, w pobliżu użytkownika, a nie na serwerze źródłowym.}.

\begin{itemize}
	\item \textbf{Język} - JavaScript/TypeScript
	\item \textbf{Kluczowe cechy} - routing, routing oparty na plikach, ESR, ISR\footnote{ISR (Interrupt Service Routine) - procedura obsługi przerwania, jest funkcją lub podprogramem w programie komputerowym wykonywanym w odpowiedzi na przerwanie. Głównym celem ISR jest obsługa przerwań, czyli sygnałów wysyłanych przez sprzęt lub oprogramowanie w celu przerwania normalnego przepływu wykonywania programu. Reagując na zakłócenia, ISR zapewniają szybką i efektywną realizację krytycznych zadań.}
	\item \textbf{Przypadki użycia} - aplikacje internetowe zoptymalizowane pod kątem SEO\footnote{SEO (search engine optimization) - proces optymalizacji strony internetowej w celu zwiększenia widoczności i ruchu z wyników wyszukiwania.}, produkty na skalę globalną.
\end{itemize}

Zalety:
\begin{itemize}
	\item oparty na React,
	\item SSR,
	\item SSG,
	\item Automatyczna optymalizacja obrazów,
	\item Oficjalne wsparcie od Vercel,
	\item Dobre SEO.
\end{itemize}

Wady:
\begin{itemize}
	\item wymaga znajomości React,
	\item mniej elastyczny od React,
	\item Konieczność dostosowania do konwencji Next.js,
	\item Mniej dostępnych komponentów niż React,
	\item Konieczność używania specyficznych rozwiązań Next.js.
\end{itemize}

\subsection{Frameworki backendowe}

Poniżej zostały przedstawione popularne frameworki backendowe w oparciu o zgromadzoną wiedzę i dostępne materiały\cite{medium_django_ruby_express_laravel}\cite{medium_spring_asp}.

\subsubsection{Django}

Django to jeden z najpopularniejszych frameworków webowych w Pythonie. Oferuje duży wachlarz solidnych funkcjonalności i przejrzysty design.


Zalety:
\begin{itemize}
	\item Architektura oparta na model-view-controller (MVC\footnote{Kontroler widoku modelu (MVC) framework to wzorzec architektoniczny, który dzieli aplikację na trzy główne komponenty logiczne: Model, Widok i Kontroler. Stąd skrót MVC. Każdy komponent architektury jest zbudowany tak, aby obsługiwać określony aspekt rozwoju aplikacji. MVC oddziela od siebie warstwę logiki biznesowej i prezentacyjną.\cite{guru99_mvc}})
	\item Bezpieczeństwo
	\item Wbudowany system ORM\footnote{ORM (Object-Relational Mapping) to technika programowania, która służy do łączenia baz danych z obiektowymi językami programowania. Umożliwia ona programistom pracę z bazami danych w sposób, który jest zgodny z ich zrozumieniem obiektowego programowania.\cite{nofluffjobs_orm}}
	\item Wbudowany interfejs administracyjny, który w łatwy sposób umożliwia zarządzaniem danymi aplikacji
\end{itemize} 


Wadami Django są przede wszystkim wydajność w przypadku bardzo dużego ruchu - wynikają bezpośrednio z zastosowania języka Python, monolityczna struktura, która utrudnia dzielenia aplikacji na mikroserwisy oraz krzywa uczenia się wynikająca z dużej ilości wbudowanych funkcjonalności.

\subsubsection{Express.js}

Express.js to minimalistyczny framework backendowy wykorzystujący javascript i Node.js


Zalety:
\begin{itemize}
	\item Lekki i nieskomplikowany - nie narzuca struktury projektu
	\item Olbrzymia społeczność
	\item Elastyczność - umożliwia budowanie zarówno bardzo prostych API jak i rozbudowanych systemów webowych z użyciem wielu bibliotek
\end{itemize}


Jedną z wad użycia Express.js jest język javascript nieposiadający silnego typowania i pełnego wachlarza funkcji, który zapewniają innego języki obiektowe. Problemem przy rozbudowanie aplikacji może również być brak narzuconej struktury - co może być problematyczne gdy na początku dobrze się jej nie zaprojektuje oraz wymaga wykorzystania wielu dodatkowych bibliotek do osiągnięcia funkcjonalności porównywalnej z innymi frameworkami.

\subsubsection{Laravel}

Laravel to framework oparty na języku PHP stawiający na prostotę i łatwość użycia.


Zalety:
\begin{itemize}
	\item Potężny silnik szablonów
	\item Prosty i intuicyjny ORM
	\item Czytelnia składnia PHP i struktura projektu
	\item Duże zasoby dokumentacji i źródeł wiedzy
\end{itemize}


Laravel zdobył popularność dzięki eleganckiej składni i bogatemu ekosystemowi. Kod jest czytelny i łatwy w utrzymaniu. Największa wadą jest zależność od wersji PHP oraz bibliotek - aktualizację często wymagają dostosowania środowiska i kodu.

\subsubsection{Spring Boot}

Spring Boot to najpopularniejszych frameworków webowych dla Javy i Kotlina. Jest nowoczesny i upraszcza tworzenie architektury opartej na mikroserwisach.


Zalety
\begin{itemize}
	\item Minimalna konfiguracja - automatyczny konfigurator upraszcza ustawianie środowiska i ustawienia frameworku.
	\item Wbudowany serwer HTTP - umożliwia uruchamianie aplikacji bezpośrednio z pliku JAR i nie wymaga dodatkowego serwera WEB
	\item Duża społeczność i dokumentacja
	\item Idealny do mikroserwisów
	\item Bogaty ekosystem integracji z bazami danych, kolejkowaniem, API itp.
\end{itemize}

Spring Boot jest dobrym rozwiązaniem do budowania backendu aplikacji webowych, jednak złożoność rośnie wraz z rozrostem projektu - mimo uproszczeń większe aplikację mogą wymagać dużej znajomości frameworku. Dodatkową wadą jest duże zużycie zasobów w porównaniu do innych frameworków.

\subsubsection{ASP.NET Core}

ASP.NET Core to wydajny framework oparty na .Net Core oraz języku programowania C#. Jest nowoczesny, opensourcowy, utrzymywany i rozwijany przez Microsoft.

Zalety
\begin{itemize}
	\item Wysoka wydajność
	\item Wsparcie Microsoftu i społeczności
	\item Wbudowane Dependecy Injection
	\item Modularna architektura
\end{itemize}


ASP.NET Core idealnie nadaję się do tworzenia wysokowydajnego backendu aplikacji webowej. Zapewnione wsparcie przez Microsoft oraz wydajność i bezpieczeństwo idealnie nadają się do projektowania skalowalnych systemów. Z wad można wyróżnić złożoną konfigurację oraz zależność od ekosystemu .NET - aktualizację frameworka mogą wymagać zmian w kodzie i środowisku.s

\subsubsection{Ruby on Rails}

Ruby on Rails często określany jako "Rails" to popularny framework opierający się na języku programowania Ruby. Filozofią Ruby on Rails jest prostota i produktywność deweloperów.


Zalety:
\begin{itemize}
	\item Możliwość szybkiego tworzenia i rozwoju aplikacji
	\item Duża liczba gotowych bibliotek
	\item Aktywna społeczność
	\item Wbudowane mechanizmy do ORM, routingu, system szablonów i API
\end{itemize}


Ruby on Rails zdobył ogromną popularność dzięki prostocie i łatwości z jaką można tworzyć aplikację webowe. Lata świetności tego frameworku przypadają na okres 2010-2015. Największą wadą jest trudność konfiguracji specyficznego środowiska uruchomieniowego oraz słaba wydajność i skalowalność dużych, złożonych aplikacji.

\subsection{Bazy danych}

\subsubsection{SQLite}

SQLite to lekkka, mała, szybka bezserwerowa relacyjna baza danych. Jej struktura składa się z jednego pliku i obsługiwana jest bezpośrednio przez aplikację - bez dodatkowego silnika bazodanowego.

Zalety:
\begin{itemize}
	\item Brak potrzeby instalacji osobnego serwera
	\item Niskie zużycie zasób
	\item Łatwość migracji
	\item Open Source
\end{itemize}

Wady:
\begin{itemize}
	\item Ograniczona skalowalność
	\item Brak zaawansowanych funkcji analitycznych
\end{itemize}

\subsubsection{PostgreSQL}

PostgreSQL to zaawansowany, wydajny system relacyjnych baz danych typu Open Source.

Zalety:
\begin{itemize}
	\item Skalowalność i wydajność
	\item Wysoka zgodność ze standardem SQL
	\item Obsługa złożonych typów oraz JSON
	\item Wsparcie procedur składniowych
\end{itemize}

Wady:
\begin{itemize}
	\item Wymaga instalacji serwera
	\item Złożoność konfiguracji
	\item Zasobożerność przy większych projektach
\end{itemize}

\subsubsection{MongoDB}

MongoDB jest nierelacyjną bazą danych NoSQL opartą na JSON. Popularna w zastosowaniach webowych i mobilnych.

Zalety:
\begin{itemize}
	\item Elastyczny schemat danych
	\item Szybkie operacje zapisu i odczytu
	\item Wysoka skalowalność pozioma
\end{itemize}

Wady:
\begin{itemize}
	\item Mniej wydajna przy złożonych zależnościach między danymi
	\item Wymaga większej kontroli nad integralnością danych
\end{itemize}

\section{Uzasadnienie wyboru Angular jako frameworka frontendowego}
\index{Uzasadnienie wyboru Angular jako frameworka frontendowego}

Angular to kompleksowe rozwiązanie do budowani aplikacji webowych. Rozwijany i wspierany przez Google, jest platformą typu "battery-included"" - czyli dostarcza wszystkie niezbędne narzędzia do tworzenia nowoczesnych aplikacji. Framework oparty jest na TypeScript co ułatwia kontrolę nad kodem.

\subsection{Kluczowe zalety}
\begin{itemize}
	\item Modułowa architektura ułatwiająca organizacje kodu,
	\item Wbudowane Dependency Injection,
	\item Jasne i dobrze opisane konwencje kodowania,
	\item Przewidywalna struktura projektu,
	\item Łatwe zarządzanie zależnościami.
\end{itemize}

\subsection{TypeScript i bezpieczeństwo}
\begin{itemize}
	\item Silne typowanie,
	\item Wczesne wykrywanie błędów podczas kompilacji,
	\item Przewidywalne zachowanie aplikacji,
	\item Lepsza dokumentacja typów i refaktoryzacja kodu w porównaniu do JavaScript.
\end{itemize}

\subsection{Komponenty i UI}
\begin{itemize}
	\item oficjalna biblioteka Material komponentów,
	\item dostępne gotowe komponenty UI,
	\item spójny wygląd i zachowanie,
	\item wbudowana responsywność,
	\item łatwe tworzenie formularzy,
	\item system szablonów.
\end{itemize}

\subsection{Zarządzanie stanem i reaktywność}
\begin{itemize}
	\item RxJS do zarządzania stanem aplikacji i operacjami asynchronicznymi,
	\item Łatwe w implementacji, rektywne formularze.
\end{itemize}

\section{Uzasadnienie wyboru ASP.NET Core jako backendu}
\index{Uzasadnienie wyboru ASP.NET Core jako backendu}

ASP.NET Core to nowoczesny, opensourcowy, wysoko wydajny framework webowy działający na platformie .NET.
Wybór został podyktowany kilkoma kluczowymi czynnikami.

\subsection{Kluczowe zalety}
\begin{itemize}
	\item Wydajność frameworku - według benchmarków jest wydajniejszy od innych popularnych frameworków\footnote{https://www.techempower.com/benchmarks/#hw=ph&test=plaintext&section=data-r23}
	\item Stabilna platforma która jest ciągle rozwijana
	\item Wieloplatformowość i elastyczność wdrożenia we wszystkich popularnych systemach operacyjnych jak i za pomocą Dockera i rozwiązań chmurowych.
\end{itemize}

\subsection{Wbudowane bezpieczeństwo}
ASP.NET Core zawiera wiele mechanizmów i technologii zapewniających bezpieczeństwo na poziomie enterpise.
\begin{itemize}
	\item CSRF
	\item XSS
	\item SQL Injection
	\item łatwa integracja z systemami JWT oraz OAuth2
\end{itemize}

\subsection{Modularna architektura oraz wstrzykiwanie zależności}
Dzięki wbudowanej dependecy injection i modularnej budowie framework jest łatwy w utrzymaniu a tworzony kod jest przejrzysty i testowalny. W prosty sposób można skalować aplikację o nowe moduły i funkcjonalności.
ASP.NET Core wspiera również wszelkie wzorce projektowe wykorzystywane w nowoczesnych aplikacjach oraz umożliwia tworzenie aplikacji z kodem C# po stronie klienta.

\subsection{Silne wsparcie Microsoft oraz społeczności}
Framework wraz z całą platformą .Net Core jest stale rozwijany i udoskonalany, do gwarantuje długotrwała stabilność i bezpieczeństwo oraz dostępność zasobów edukacyjnych.

Ponad wyżej wymienione zalety posiadam wieloletnie doświadczenie w wykorzystaniu frameworku co również wpłynęło na jego wybór jako backend projektowanego systemu.

\section{Uzasadnienie wyboru SQLite jako bazy danych}
\index{Uzasadnienie wyboru SQLite jako bazy danych}

SQLite stał się wyborem jako bazadanych w projektowanym systemie ze względu na prostotę, lekkość i brak potrzeb dodatkowych konfiguracji serwera i narzutu zasobów z tym związanych. Jest to idealne rozwiązanie do aplikacji webowych i mobilnych, szybkiego projektowania i rozwoju struktury danych oraz wdrażania programów na systemach o ograniczonych zasobach sprzętowych.

Dzięki przechowywaniu danych w jednym pliku SQLite zapewnia prostotę wdrożenia i przenoszenia baz danych - co wiąże się również dość prostym backupem bazy. Zgodność z SQL gwarantuje spójność danych mimo braku serwera.

\section{Omówienie innych wykorzystanych technologii}
\index{Omówienie innych wykorzystanych technologii}

\subsection{Angular Material}

Jest to oficjalna biblioteka komponentów UI dla Angular, która jest zgodna z Material Design\footnote{Material Design 3 to system projektowania Google typu open source do tworzenia pięknych, użytecznych produktów. Z ang. \cite{material_design}}.
Komponenty są zgodne ze standardem dostępności WCAG 2.1, responsywne i zapewniają jednolity wygląd całej aplikacji.

\subsubsection{Zalety w kontekście projektu}
\begin{itemize}
	\item spójny wygląd systemu,
	\item duża ilość gotowych komponentów,
	\item łatwość implementacji,
	\item responsywne,
	\item możliwość konfiguracji wyglądu.
\end{itemize}

\subsection{JWT (JSON Web Tokens)}

Tokeny JWT wykorzystywane są do autoryzacji użytkowników. Gdy użytkownik zostanie zalogowany, każde następne żądanie będzie zawierać token JWT, dzięki czemu uzyska on dostęp do zasobów przewidzianych dla zalogowanych użytkowników lub zablokowanych za odpowiednią rolą (administrator, moderator).

\subsubsection{Wykorzystanie w projekcie}

\begin{itemize}
	\item uwierzytelnianie,
	\item zarządzanie sesją,
	\item autoryzacja.
\end{itemize}

\subsection{Leaflet.js}

\subsection{Git i GitHub}

Do śledzenia zmian kodu, wersjonowania oraz dla bezpieczeństwa projekt kod i wszystkie niezbędne pliki są przechowywane w systemie Git. Repozytorium Git znajduję się w serwisie Github\footnote{https://github.com/}.

Git to rozproszony system kontroli wersji stworzony przez Linus'a Torvalds'a. Jest to aktualnie najczęściej wykorzystywane narzędzie do wersjonowania kodu na świecie, zarówno wśród hobbystów i pasjonatów programowania jak i w środowiskach profesjonalnych i biznesach.

\subsection{JetBrains IDE}

Do pracy nad systemem wykorzystywane są narzędzia firmy JetBrains. Jetbrains IDE\footnote{Zintegrowane środowisko programistyczne, IDE – program lub zespół programów służących do tworzenia, modyfikowania, testowania i konserwacji oprogramowania.\cite{wikipedia.pl}} WebStorm dla prac Frontendowych Angular oraz IDE Rider do pracy z kodem C# i Asp.Net Core. Są to zaawansowane narzędzia developerskie zapewniające kompleksowe wsparcie dla wybranych technologii.