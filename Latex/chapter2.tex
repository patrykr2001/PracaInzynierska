%%%%%%%%%%%%%%%%%%%%%%%%%%%%%%%%%%%%%%%%%
% Szablon pracy dyplomowej
% Wydział Informatyki 
% Zachodniopomorski Uniwersytet Technologiczny w Szczecinie
% autor Joanna Kołodziejczyk (jkolodziejczyk@zut.edu.pl)
% Bardzo wczesnym pierwowzorem szablonu był
% The Legrand Orange Book
% Version 5.0 (29/05/2025)
%
% Modifications to LOB assigned by %JK
%%%%%%%%%%%%%%%%%%%%%%%%%%%%%%%%%%%%%%%%%


%----------------------------------------------------------------------------------------
%	CHAPTER 2
%----------------------------------------------------------------------------------------
\chapter{Wybór i uzasadnienie technologii}
\label{rozdzial2}

W tej sekcji zostało przedstawione przegląd i porównanie dostępnych technologii webowych oraz uzasadniony ich wybór w implementacji systemu.

\section{Analiza dostępnych rozwiązań technologicznych}
\index{Analiza dostępnych rozwiązań technologicznych}

\subsection{Frameworki frontendowe}
Poniżej zostało przedstawione porównanie głównych frameworków frontendowych.\cite{reactvsvuevsangularvsnextjs}

\subsubsection{Angular}
Kompletny framework utrzymywany przez Google. Angular wymusza strukture z dużym naciskiem na TypeScript, RxJS i skalowalną architekturę.
\begin{itemize}
	\item \textbf{Język} - TypeScript,
	\item \textbf{Kluczowe cechy} - wstrzykiwanie zależności\footnote{Wstrzykiwanie zależności – wzorzec projektowy i wzorzec architektury oprogramowania polegający na usuwaniu bezpośrednich zależności pomiędzy komponentami na rzecz architektury typu plug-in. Polega na przekazywaniu gotowych, utworzonych instancji obiektów udostępniających swoje metody i właściwości obiektom, które z nich korzystają.}, RxJS\footnote{RxJS to biblioteka do programowania reaktywnego przy użyciu Observables, aby ułatwić tworzenie kodu asynchronicznego lub opartego na wywołaniach zwrotnych. Z ang. \cite{rxjs.dev}}, routing, testowanie, CLI\footnote{CLI to skrót od „Command Line Interface” (interfejs wiersza poleceń). Jest to sposób interakcji użytkownika z komputerem poprzez wprowadzanie poleceń tekstowych w terminalu lub wierszu poleceń, zamiast korzystania z interfejsu graficznego (GUI).},
	\item \textbf{Przypadki użycia} - Aplikacje dla przedsiębiorstw, narzędzia wewnętrzne, sektory regulowane.
\end{itemize}

Zalety:
\begin{itemize}
	\item Kompleksowe rozwiązanie,
	\item Wbudowany system zarządzania stanem (RxJS),
	\item Sile typowanie dzięki wykorzystaniu TypeScript,
	\item Oficjalna biblioteka komponentów (Material),
	\item Dependency injection wbudowany w framework,
	\item Oficjalne wsparcie od Google,
	\item Długoterminowe wsparcie (LTS),
	\item Bogata dokumentacja i duża społeczność.
\end{itemize}

Wady:
\begin{itemize}
	\item Wysoki próg wejścia,
	\item Większy rozmiar aplikacji w porównaniu do innych frameworków,
	\item Mniej elastyczny niz React,
	\item Wymaga znajomości wielu konceptów inżynierii oprogramowania,
	\item Mniejsza ilość dostępnych komponentów niż w React.
\end{itemize}

\subsubsection{React}
Biblioteka UI opracowana przez firmę Meta. React skupia się wyłącznie na warstwie widoku, dając elastyczność w wyborze własnych narzędzi do routingu, zarządzania stanem i budowania.

\begin{itemize}
	\item \textbf{Język} - JavaScript lub TypeScript (z JSX\footnote{JSX (JavaScript XML) – format zapisu kodu HTML oraz XML wewnątrz języka JavaScript. Pierwotnie zaproponowany przez Facebooka i użyty w bibliotece React.js. Korzystają z niego także inne biblioteki.}),
	\item \textbf{Kluczowe cechy} - hooki, wirtualny DOM\footnote{Obiektowy model dokumentu – sposób reprezentacji złożonych dokumentów XML i HTML w postaci modelu obiektowego. Model ten jest niezależny od platformy i języka programowania.}, architektura oparta na komponentach,
	\item \textbf{Przypadki użycia} - wszelkiego rodzaju interfejsy użytkownika, od prostych komponentów po aplikacje wieloplatformowe.
\end{itemize}

Zalety:
\begin{itemize}
	\item Elastyczność i swoboda w wyborze narzędzi,
	\item Duża ilość dostępnych komponentów i bibliotek,
	\item Mniejszy próg wejścia i szybsza nauka,
	\item Virtual DOM dla lepszej wydajności,
	\item Popularny z dużą społecznością,
	\item Wsparcie od Meta (Facebook).
\end{itemize}

Wady:
\begin{itemize}
	\item Wymaga dodatkowych bibliotek do pełnej funkcjonalności,
	\item Brak wbudowanego systemu zarządzania stanem,
	\item Konieczność samodzielnego wyboru narzędzi.
\end{itemize}

\subsubsection{Vue}
Progresywny framework opracowany przez Evana You. Vue kładzie nacisk na prostotę i wydajność, w wbudowaną mocą reaktywności i kompozycji.

\begin{itemize}
	\item \textbf{Język} - JavaScript/TypeScript z opartymi na HTML\footnote{HTML – język znaczników stosowany do tworzenia dokumentów hipertekstowych.} szablonami,
	\item \textbf{Kluczowe cechy} - API kompozycji, SFC\footnote{Vue Single-File Components(SFC) - Komponenty jednoplikowe Vue}, silnik reaktywności,
	\item \textbf{Przypadki użycia} - pulpity nawigacyjne, panele administracyjne, CMS-y\footnote{Content Management System (CMS) - System zarządzania treścią – oprogramowanie pozwalające na łatwe utworzenie i prowadzenie serwisu WWW, a także jego późniejszą aktualizację i rozbudowę, również przez redakcyjny personel nietechniczny.} i SPA\footnote{Single Page Application (SPA) - jednostronicowa aplikacja internetowa. Rodzaj aplikacji webowej, która działa w przeglądarce użytkownika i nie wymaga przeładowywania strony przy przechodzeniu między sekcjami}.
\end{itemize}

Zalety:
\begin{itemize}
	\item Niski próg wejścia,
	\item Elastyczność implementacji,
	\item Bardzo dobra dokumentacja,
	\item Mniejszy rozmiar aplikacji,
	\item Prosty system komponentów.
\end{itemize}

Wady:
\begin{itemize}
	\item Mniejsza społeczność niż w przypadku innych frameworków,
	\item Mniej dostępnych komponentów,
	\item Mniej popularny w profesjonalnych zastosowaniach,
	\item Brak oficjalnego wsparcia dużej firmy.
\end{itemize}

\subsubsection{Next.js}
Oparty na React framework full-stack zbudowany przez firmę Vercel. Next.js przekształca Reacta w platformę do budowania aplikacji SSR\footnote{Server Side Rendering (SSR) - renderowanie po stronie serwera, to technika generowania strony internetowej na serwerze jeszcze przed wysłaniem jej do klienta. Jest to alternatywa dla Client Side Rendering (CSR)}, SSG\footnote{SSG (Static Site Generation) - technika tworzenia stron internetowych, gdzie całość zawartości strony generowana jest w momencie jej budowy, a następnie serwowana jako statyczne pliki HTML.} oraz ESR\footnote{Edge Side Rendering (ESR) - technika renderowania stron internetowych, która polega na wstępnym renderowaniu stron internetowych na obrzeżach sieci, w pobliżu użytkownika, a nie na serwerze źródłowym.}.

\begin{itemize}
	\item \textbf{Język} - JavaScript/TypeScript
	\item \textbf{Kluczowe cechy} - routing, routing oparty na plikach, ESR, ISR\footnote{ISR (Interrupt Service Routine) - procedura obsługi przerwania, jest funkcją lub podprogramem w programie komputerowym wykonywanym w odpowiedzi na przerwanie. Głównym celem ISR jest obsługa przerwań, czyli sygnałów wysyłanych przez sprzęt lub oprogramowanie w celu przerwania normalnego przepływu wykonywania programu. Reagując na zakłócenia, ISR zapewniają szybką i efektywną realizację krytycznych zadań.}
	\item \textbf{Przypadki użycia} - aplikacje internetowe zoptymalizowane pod kątem SEO\footnote{SEO (search engine optimization) - proces optymalizacji strony internetowej w celu zwiększenia widoczności i ruchu z wyników wyszukiwania.}, produkty na skalę globalną.
\end{itemize}

Zalety:
\begin{itemize}
	\item oparty na React,
	\item SSR,
	\item SSG,
	\item Automatyczna optymalizacja obrazów,
	\item Oficjalne wsparcie od Vercel,
	\item Dobre SEO.
\end{itemize}

Wady:
\begin{itemize}
	\item wymaga znajomości React,
	\item mniej elastyczny od React,
	\item Konieczność dostosowania do konwencji Next.js,
	\item Mniej dostępnych komponentów niż React,
	\item Konieczność używania specyficznych rozwiązań Next.js.
\end{itemize}

\subsection{Frameworki backendowe}

\subsection{Bazy danych}

\section{Uzasadnienie wyboru Angular jako frameworka frontendowego}
\index{Uzasadnienie wyboru Angular jako frameworka frontendowego}

Angular to kompleksowe rozwiązanie do budowani aplikacji webowych. Rozwijany i wspierany przez Google, jest platformą typu "battery-included" - czyli dostarcza wszystkie niezbędne narzędzia do tworzenia nowoczesnych aplikacji. Framework oparty jest na TypeScript co ułatwia kontrolę nad kodem.

\subsection{Kluczowe zalety}
\begin{itemize}
	\item Modułowa architektura ułatwiająca organizacje kodu,
	\item Wbudowane Dependency Injection,
	\item Jasne i dobrze opisane konwencje kodowania,
	\item Przewidywalna struktura projektu,
	\item Łatwe zarządzanie zależnościami.
\end{itemize}

\subsection{TypeScript i bezpieczeństwo}
\begin{itemize}
	\item Silne typowanie,
	\item Wczesne wykrywanie błędów podczas kompilacji,
	\item Przewidywalne zachowanie aplikacji,
	\item Lepsza dokumentacja typów i refaktoryzacja kodu w porównaniu do JavaScript.
\end{itemize}

\subsection{Komponenty i UI}
\begin{itemize}
	\item oficjalna biblioteka Material komponentów,
	\item dostępne gotowe komponenty UI,
	\item spójny wygląd i zachowanie,
	\item wbudowana responsywność,
	\item łatwe tworzenie formularzy,
	\item system szablonów.
\end{itemize}

\subsection{Zarządzanie stanem i reaktywność}
\begin{itemize}
	\item RxJS do zarządzania stanem aplikacji i operacjami asynchronicznymi,
	\item Łatwe w implementacji, rektywne formularze.
\end{itemize}

\section{Uzasadnienie wyboru ASP.NET Core jako backendu}
\index{Uzasadnienie wyboru ASP.NET Core jako backendu}

\section{Uzasadnienie wyboru SQLite jako bazy danych}
\index{Uzasadnienie wyboru SQLite jako bazy danych}

\section{Omówienie innych wykorzystanych technologii}
\index{Omówienie innych wykorzystanych technologii}

\subsection{Angular Material}

\subsection{JWT}

\subsection{Leaflet.js}

\subsection{Git i GitHub}

\subsection{JetBrains IDE}

%  (Leaflet, Material Design)
