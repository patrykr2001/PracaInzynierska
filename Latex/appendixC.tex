%%%%%%%%%%%%%%%%%%%%%%%%%%%%%%%%%%%%%%%%%
% Szablon pracy dyplomowej
% Wydział Informatyki 
% Zachodniopomorski Uniwersytet Technologiczny w Szczecinie
% autor Joanna Kołodziejczyk (jkolodziejczyk@zut.edu.pl)
% Bardzo wczesnym pierwowzorem szablonu był
% The Legrand Orange Book
% Version 5.0 (29/05/2025)2023)
%
% Modifications to LOB assigned by %JK
%%%%%%%%%%%%%%%%%%%%%%%%%%%%%%%%%%%%%%%%%

\chapter{Komponent mapy}
\label{chapter:dodatek_C}

\textit{Poniżej przedstawiono fragmenty kodu dotyczące komponentu mapy.}

\begin{lstlisting}[style=tsstyle, caption={Inicjalizacja},label={lst:inicjalizacjamapy}]
	private initMap(): void {
		// Sprawdzamy, czy mapa już istnieje
		if (this.map) {
			this.map.remove();
		}
		// Domyślna lokalizacja (centrum Polski)
		const defaultLocation = { lat: 51.9194, lng: 19.1451 };
		const initialLocation = this.initialLocation || defaultLocation;
		// Inicjalizacja mapy
		this.map = L.map('map', {
			maxBounds: this.POLAND_BOUNDS,
			maxBoundsViscosity: 1.0,
			minZoom: 6,
			maxZoom: 18
		}).setView([initialLocation.lat, initialLocation.lng], this.initialZoom);
		// Dodanie warstwy OpenStreetMap
		L.tileLayer('https://{s}.tile.openstreetmap.org/
			{z}/{x}/{y}.png', {
			attribution: '© OpenStreetMap contributors'
		}).addTo(this.map);
		// Ograniczenie widoku do granic Polski
		this.map.setMaxBounds(this.POLAND_BOUNDS);
		// Dodanie początkowego markera tylko jeśli przekazano initialLocation
		if (this.initialLocation) {
			this.marker = L.marker([this.initialLocation.lat, this.initialLocation.lng], { icon: this.defaultIcon }).addTo(this.map);
		}
		// Dodanie obsługi kliknięcia na mapę tylko jeśli allowSelection jest true
		if (this.allowSelection) {
			this.map.on('click', (e: L.LeafletMouseEvent) => {
				const lat = e.latlng.lat;
				const lng = e.latlng.lng;
				// Usunięcie poprzedniego markera
				if (this.marker) {
					this.map?.removeLayer(this.marker);
				}
				// Dodanie nowego markera
				this.marker = L.marker([lat, lng], { icon: this.defaultIcon }).addTo(this.map!);
				// Emisja wybranej lokalizacji
				this.locationSelected.emit({ lat, lng });
			});
		}
		// Dodanie markerów obserwacji po inicjalizacji mapy
		this.updateObservationMarkers();
		// Odświeżenie mapy po załadowaniu
		setTimeout(() => {
			this.map?.invalidateSize();
		}, 200);
	}
\end{lstlisting}

\begin{lstlisting}[language=CSS, caption={Style CSS mapy},label={lst:cssmapy}]
.map-container {
	height: 100%;
	min-height: 400px;
	width: 100%;
	position: relative;
	border-radius: 4px;
	overflow: hidden;
	margin-bottom: 1rem;
}

:host {
	display: block;
	width: 100%;
	height: 100%;
}

// Responsywne style dla mapy
@media (min-width: 1200px) {
	.map-container {
		min-height: 600px;
	}
}

@media (min-width: 1600px) {
	.map-container {
		min-height: 800px;
	}
}

@media (min-width: 1920px) {
	.map-container {
		min-height: 900px;
	}
}

// Style dla mapy w formularzach (mniejsza wysokość)
.map-container.form-map {
	min-height: 300px;
	max-height: 400px;
}

// Style dla mapy na stronie głównej (większa wysokość)
.map-container.home-map {
	min-height: 500px;
}
\end{lstlisting}

\begin{lstlisting}[style=csharp, caption={Przykład użycia walidacji danych geograficznych},label={lst:backendwalidacjageograficzna}]
public async Task<BirdObservationDto> CreateObservationAsync(CreateBirdObservationDto observationDto, string userId)
{
	// Sprawdź czy ptak istnieje
	var bird = await _context.Birds.FindAsync(observationDto.BirdId);
	if (bird == null)
	throw new ArgumentException($"Ptak o ID {observationDto.BirdId} nie istnieje.");
	
	// Sprawdź czy użytkownik istnieje
	var user = await _context.Users.FindAsync(userId);
	if (user == null)
	throw new ArgumentException($"Użytkownik o ID {userId} nie istnieje.");
	
	var (isLatitudeValid, latitude) = ValidateAndParseCoordinate(observationDto.Latitude, true);
	if (!isLatitudeValid)
	throw new ArgumentException("Nieprawidłowa szerokość geograficzna. Wartość musi być liczbą z zakresu -90 do 90.");
	
	var (isLongitudeValid, longitude) = ValidateAndParseCoordinate(observationDto.Longitude, false);
	if (!isLongitudeValid)
	throw new ArgumentException("Nieprawidłowa długość geograficzna. Wartość musi być liczbą z zakresu -180 do 180.");
	
	var observation = new BirdObservation
	{
		BirdId = observationDto.BirdId,
		UserId = userId,
		Latitude = latitude.Value,
		Longitude = longitude.Value,
		ObservationDate = observationDto.ObservationDate,
		Description = observationDto.Description,
		NumberOfBirds = observationDto.NumberOfBirds,
		WeatherConditions = observationDto.WeatherConditions,
		Habitat = observationDto.Habitat,
		IsVerified = false
	};
	
	// Dodanie obrazów do obserwacji
	if (observationDto.Images != null && observationDto.Images.Any())
	{
		foreach (var image in observationDto.Images)
		{
			var imageUrl = await SaveImageAsync(image);
			observation.ImageUrls.Add(imageUrl);
		}
	}
	
	_context.BirdObservations.Add(observation);
	await _context.SaveChangesAsync();
	
	return await GetObservationByIdAsync(observation.Id) ?? throw new Exception("Failed to create observation");
}
\end{lstlisting}

\pagebreak

\begin{lstlisting}[style=csharp, caption={Implementacja serwisu tygodni},label={lst:backendserwistygodni}]
public async Task<List<object>> GetObservationWeeksAsync(int? year = null)
{
	var observations = await _context.BirdObservations.ToListAsync();
	if (observations.Count == 0) return new List<object>();
	if (year.HasValue)
	{
		observations = observations.Where(o => o.ObservationDate.Year == year.Value).ToList();
	}
	if (observations.Count == 0) return new List<object>();
	var dates = observations.Select(o => o.ObservationDate.Date).ToList();
	var minDate = dates.Min();
	var maxDate = dates.Max();
	
	var startOfWeek = new Func<DateTime, DateTime>(d => 
	d.AddDays(-(d.DayOfWeek == DayOfWeek.Sunday ? 6 : ((int)d.DayOfWeek - 1))).Date);
	
	var current = startOfWeek(minDate);
	var end = startOfWeek(maxDate);
	var result = new List<object>();
	
	while (current <= end)
	{
		var weekStart = current;
		var weekEnd = current.AddDays(6);
		var count = observations.Count(o => 
		o.ObservationDate.Date >= weekStart && o.ObservationDate.Date <= weekEnd);
		
		if (count > 0)
		{
			result.Add(new {
				start = weekStart,
				end = weekEnd,
				count
			});
		}
		current = current.AddDays(7);
	}
	return result;
}
\end{lstlisting}