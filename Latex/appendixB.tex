%%%%%%%%%%%%%%%%%%%%%%%%%%%%%%%%%%%%%%%%%
% Szablon pracy dyplomowej
% Wydział Informatyki 
% Zachodniopomorski Uniwersytet Technologiczny w Szczecinie
% autor Joanna Kołodziejczyk (jkolodziejczyk@zut.edu.pl)
% Bardzo wczesnym pierwowzorem szablonu był
% The Legrand Orange Book
% Version 5.0 (29/05/2025)2023)
%
% Modifications to LOB assigned by %JK
%%%%%%%%%%%%%%%%%%%%%%%%%%%%%%%%%%%%%%%%%

\chapter{System autentykacji i autoryzacji}
\label{chapter:dodatek_B}

\textit{Poniżej przedstawiono fragmenty kodu dotyczące systemu autentykacji i autoryzacji.}

\begin{lstlisting}[style=csharp, caption={Implementacja logowania},label={lst:logowanie}]
	public async Task<AuthResponseDto> LoginAsync(LoginDto loginDto)
	{
		var user = await _userManager.FindByEmailAsync(loginDto.Email);
		if (user == null)
		{
			throw new InvalidOperationException("Nieprawidłowy email lub hasło.");
		}
		
		if (await _userManager.IsLockedOutAsync(user))
		{
			var lockoutEnd = await _userManager.GetLockoutEndDateAsync(user);
			throw new InvalidOperationException($"Konto jest zablokowane do {lockoutEnd}.");
		}
		
		var result = await _signInManager.CheckPasswordSignInAsync(user, loginDto.Password, true);
		if (!result.Succeeded)
		{
			if (result.IsLockedOut)
			{
				throw new InvalidOperationException($"Konto zostało zablokowane na {LOCKOUT_DURATION_MINUTES} minut z powodu zbyt wielu nieudanych prób logowania.");
			}
			throw new InvalidOperationException("Nieprawidłowy email lub hasło.");
		}
		
		await _userManager.ResetAccessFailedCountAsync(user);
		
		var roles = await _userManager.GetRolesAsync(user);
		var token = GenerateAccessToken(user, roles);
		var refreshToken = GenerateRefreshToken();
		
		user.RefreshToken = refreshToken;
		user.RefreshTokenExpiryTime = DateTime.UtcNow.AddDays(
			REFRESH_TOKEN_EXPIRATION_DAYS);
		await _userManager.UpdateAsync(user);
		
		return new AuthResponseDto
		{
			AccessToken = token,
			RefreshToken = refreshToken,
			User = new UserDto
			{
				Id = user.Id,
				Username = user.UserName!,
				Email = user.Email!,
				Roles = roles.ToArray()
			}
		};
	}
\end{lstlisting}

\begin{lstlisting}[style=csharp, caption={Implementacja rejestracji},label={lst:rejestracja}]
	public async Task<AuthResponseDto> RegisterAsync(RegisterDto registerDto)
	{
		if (registerDto.Password != registerDto.ConfirmPassword)
		{
			throw new InvalidOperationException("Hasła nie są identyczne.");
		}
		
		var passwordValidator = new PasswordValidator<ApplicationUser>();
		var result = await passwordValidator.ValidateAsync(_userManager, null!, registerDto.Password);
		if (!result.Succeeded)
		{
			throw new InvalidOperationException(string.Join(", ", result.Errors.Select(e => e.Description)));
		}
		
		var user = new ApplicationUser
		{
			UserName = registerDto.Email,
			Email = registerDto.Email,
			CreatedAt = DateTime.UtcNow,
			EmailConfirmed = true
		};
		
		var createResult = await _userManager.CreateAsync(user, registerDto.Password);
		if (!createResult.Succeeded)
		{
			throw new InvalidOperationException(string.Join(", ", createResult.Errors.Select(e => e.Description)));
		}
		
		await _userManager.AddToRoleAsync(user, AuthorizationConstants.UserRole);
		
		var roles = await _userManager.GetRolesAsync(user);
		var token = GenerateAccessToken(user, roles);
		var refreshToken = GenerateRefreshToken();
		
		user.RefreshToken = refreshToken;
		user.RefreshTokenExpiryTime = DateTime.UtcNow.AddDays(
			REFRESH_TOKEN_EXPIRATION_DAYS);
		await _userManager.UpdateAsync(user);
		
		return new AuthResponseDto
		{
			AccessToken = token,
			RefreshToken = refreshToken,
			User = new UserDto
			{
				Id = user.Id,
				Username = user.UserName!,
				Email = user.Email!,
				Roles = new[] { AuthorizationConstants.UserRole }
			}
		};
	}
\end{lstlisting}

\pagebreak

\begin{lstlisting}[style=csharp, caption={Kontroler autentykacji},label={lst:kontrolerautentykacji}]
[ApiController]
[Route("api/[controller]")]
public class AuthController : ControllerBase
{
	private readonly IAuthService _authService;
	
	public AuthController(IAuthService authService)
	{
		_authService = authService;
	}
	
	[HttpPost("login")]
	public async Task<ActionResult<AuthResponseDto>> Login([FromBody] LoginDto loginDto)
	{
		if (!ModelState.IsValid)
		{
			return BadRequest(new { 
				message = "Błąd walidacji",
				errors = ModelState.Values
				.SelectMany(v => v.Errors)
				.Select(e => e.ErrorMessage)
			});
		}
		
		try
		{
			var response = await _authService.LoginAsync(loginDto);
			return Ok(response);
		}
		catch (InvalidOperationException ex)
		{
			return BadRequest(new { message = ex.Message });
		}
	}
	
	[HttpPost("register")]
	public async Task<ActionResult<AuthResponseDto>> Register([FromBody] RegisterDto registerDto)
	{
		// Implementacja rejestracji
	}
	
	[HttpPost("refresh-token")]
	public async Task<ActionResult<AuthResponseDto>> RefreshToken([FromBody] RefreshTokenDto refreshTokenDto)
	{
		// Implementacja odświeżania tokenu
	}
	
	[Authorize]
	[HttpPost("logout")]
	public async Task<IActionResult> Logout()
	{
		var userId = User.FindFirst(ClaimTypes.NameIdentifier)?.Value;
		if (userId != null)
		{
			await _authService.RevokeRefreshTokenAsync(userId);
		}
		return Ok(new { message = "Wylogowano pomyślnie." });
	}
}
\end{lstlisting}

\begin{lstlisting}[style=tsstyle, caption={Serwis autentykacji po stronie frontendu},label={lst:serwisautentykacjifrontend}]
@Injectable({
	providedIn: 'root'
})
export class AuthService {
	private baseUrl = `${environment.api.baseUrl}${environment.api
			.endpoints.auth}`;
	private currentUserSubject = new BehaviorSubject<User | null>(null);
	currentUser$ = this.currentUserSubject.asObservable();
	private tokenRefreshTimeout: any;
	private isRefreshing = false;
	private refreshTokenSubject = new BehaviorSubject<string | null>(null);
	
	constructor(
	private http: HttpClient,
	private router: Router
	) {
		const savedUser = localStorage.getItem('currentUser');
		if (savedUser) {
			this.currentUserSubject.next(JSON.parse(savedUser));
			this.setupTokenRefresh();
		}
	}
	
	private setupTokenRefresh(): void {
		const refreshToken = localStorage.getItem('refreshToken');
		if (refreshToken) {
			this.tokenRefreshTimeout = setTimeout(() => {
				this.refreshToken().subscribe({
					error: () => {
						this.handleAuthError();
					}
				});
			}, 14 * 60 * 1000);
		}
	}
	
	refreshToken(): Observable<AuthResponse> {
		const refreshToken = localStorage.getItem('refreshToken');
		if (!refreshToken) {
			return throwError(() => new Error('No refresh token available'));
		}
		
		if (this.isRefreshing) {
			return this.refreshTokenSubject.pipe(
			switchMap(token => {
				if (token) {
					return of({ 
						accessToken: token, 
						refreshToken: '',
						user: this.currentUserSubject.value || {
							id: '0',
							email: '',
							username: '',
							role: UserRole.User,
							roles: [UserRole.User]
						}
					});
				}
				return throwError(() => new Error('Token refresh failed'));
			})
			);
		}
		
		this.isRefreshing = true;
		this.refreshTokenSubject.next(null);
		
		return this.http.post<AuthResponse>(`${this.baseUrl}
		${environment.api.endpoints.authEndpoints
			.refreshToken}`, { refreshToken })
		.pipe(
		tap(response => {
			this.setTokens(response);
			this.refreshTokenSubject.next(response.accessToken);
			this.isRefreshing = false;
		}),
		catchError(error => {
			this.isRefreshing = false;
			this.refreshTokenSubject.next(null);
			return throwError(() => error);
		})
		);
	}
	
	private handleAuthError(): void {
		this.logout();
		this.router.navigate(['/login'], {
			queryParams: { returnUrl: this.router.url }
		});
	}
	
	private setTokens(response: AuthResponse): void {
		localStorage.setItem('accessToken', response.accessToken);
		localStorage.setItem('refreshToken', response.refreshToken);
		localStorage.setItem('currentUser', JSON.stringify(response.user));
		this.currentUserSubject.next(response.user);
	}
	
	updateCurrentUser(user: User): void {
		localStorage.setItem('currentUser', JSON.stringify(user));
		this.currentUserSubject.next(user);
	}
	
	login(loginDto: LoginDto): Observable<AuthResponse> {
		return this.http.post<AuthResponse>(`${this.baseUrl}
			${environment.api.endpoints.authEndpoints.login}`, loginDto)
		.pipe(
		tap(response => {
			this.setTokens(response);
			this.setupTokenRefresh();
		}),
		catchError(this.handleError.bind(this))
		);
	}
	
	register(registerDto: RegisterDto): Observable<AuthResponse> {
		return this.http.post<AuthResponse>(`${this.baseUrl}
			${environment.api.endpoints.authEndpoints
				.register}`, registerDto)
		.pipe(
		tap(response => {
			this.setTokens(response);
			this.setupTokenRefresh();
		}),
		catchError(this.handleError.bind(this))
		);
	}
	
	logout(): Observable<void> {
		const refreshToken = localStorage.getItem('refreshToken');
		if (refreshToken) {
			return this.http.post<void>(`${this.baseUrl}
			${environment.api.endpoints.authEndpoints
					.revokeToken}`, { refreshToken })
			.pipe(
			finalize(() => {
				this.clearAuthData();
			})
			);
		}
		this.clearAuthData();
		return new Observable(subscriber => {
			subscriber.next();
			subscriber.complete();
		});
	}
	
	getCurrentUser(): User | null {
		return this.currentUserSubject.value;
	}
	
	isLoggedIn(): boolean {
		return !!this.currentUserSubject.value;
	}
	
	isAdmin(): boolean {
		const user = this.currentUserSubject.value;
		return user?.roles?.includes(UserRole.Admin) ?? false;
	}
	
	getToken(): string | null {
		return localStorage.getItem('accessToken');
	}
	
	private handleError(error: HttpErrorResponse) {
		let errorMessage = 'Wystąpił błąd podczas operacji';
		if (error.error instanceof ErrorEvent) {
			errorMessage = `Błąd: ${error.error.message}`;
		} else {
			errorMessage = `Kod błędu: ${error.status}, wiadomość: ${error.error.message || errorMessage}`;
		}
		return throwError(() => new Error(errorMessage));
	}
	
	private clearAuthData(): void {
		localStorage.removeItem('accessToken');
		localStorage.removeItem('refreshToken');
		localStorage.removeItem('currentUser');
		this.currentUserSubject.next(null);
		if (this.tokenRefreshTimeout) {
			clearTimeout(this.tokenRefreshTimeout);
		}
	}
}
\end{lstlisting}

\begin{lstlisting}[style=tsstyle, caption={Interceptor HTTP},label={lst:interceptorhttp}]
	export const authInterceptor: HttpInterceptorFn = (req, next) => {
		const router = inject(Router);
		const authService = inject(AuthService);
		const token = localStorage.getItem('accessToken');
		
		if (token) {
			const clonedReq = req.clone({
				headers: req.headers.set('Authorization', `Bearer ${token}`)
			});
			req = clonedReq;
		}
		
		return next(req).pipe(
		catchError((error: HttpErrorResponse) => {
			if (error.status === 401 && !req.url.includes('/refresh-token')) {
				return authService.refreshToken().pipe(
				switchMap(response => {
					const newReq = req.clone({
						headers: req.headers.set('Authorization', `Bearer ${response.accessToken}`)
					});
					return next(newReq);
				}),
				catchError(refreshError => {
					authService.logout().subscribe(() => {
						router.navigate(['/login'], {
							queryParams: { returnUrl: router.url }
						});
					});
					return throwError(() => refreshError);
				})
				);
			}
			
			if (error.status === 403) {
				router.navigate(['/login'], { 
					queryParams: { returnUrl: router.url }
				});
			}
			
			return throwError(() => error);
		})
		);
	};
\end{lstlisting}